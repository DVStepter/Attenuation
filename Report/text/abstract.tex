The ability to detect and localize radioactive sources in an urban environment is critical to national security. A wide-range of detection systems have demonstrated the ability to detect the presence of radioactive materials through walls and localize by producing images of the comparative gamma flux. Localization is merely visually locating the "hot spot" in the image, which can be greatly effected by gamma scattering in walls. This inaccurate localization requires search and recovery personnel to conduct a broader search, increasing their exposure time and accumulated dose, instead of moving directly to the material of interest. Additionally, the extended search duration provides hostile actors time to move, detonate, or take any other action to impede recovery operations. The ability to quantify (i.e. determine the exact flux and energy) the flux through building surfaces could allow search and recovery personnel to externally detect, identify (by energy), and localize (by focusing on the target energy) sources. This paper examines how quantification can provide highly accurate source localization through materials. Specifically, a model was developed that would produce a quantified surface flux from a source within a building. With the quantified surface flux simple image processing techniques were applied to provide highly accurate source localization.
