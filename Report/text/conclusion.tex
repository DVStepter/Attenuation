\noindent This study developed a method to artificially create a quantified surface flux to examine its utility. Furthermore, this study developed an easily implementable and accurate localization method as one of the potential benefits of gamma quantification.

\subsection{Limitations}
\noindent One main limitation of this study is that it exists in a perfect world where the user has 100$\%$ accurate information. There is no uncertainty in position, time, energy or any measurable metric. In reality a UAS will have some GPS inaccuracy, a detector system will not be 100$\%$ efficient, and may have less than optimal energy resolution. All of these factors combined may reduce the feasibility and/or utility of quantification. Additionally, these simulations were run generating six million particles. For high activity sources this can occur in a matter of seconds, but for lower activity sources, the time required to produce (and detect) that many particles may be too long in a search and recovery scenario.

\subsection{Recommended Future Work}
\noindent In the future, more complexity should be added to the model to include background radiation, more complex geometry, and true detector performance. Additionally, this model should be modified to represent a real location that can be used to verify results.

All geometry, simulation, parsing codes, analysis code, and perspective images for all simulations can be viewed at https://github.com/DVStepter/Attenuation.
