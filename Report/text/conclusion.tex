\noindent This study developed a method to artificially create a quantified surface flux to examine its utility. Furthermore, this study developed an easily implementable and accurate localization method as one of the potential benefits of gamma quantification. Segregating energy and localizing on full energy deposition proved to to be highly accurate.

\subsection{Limitations}
\noindent One main limitation of this study is that it exists in a perfect world where the user has 100$\%$ accurate information. There is no uncertainty in position, time, energy or any measurable metric. In reality, a UAS will have some GPS inaccuracy, a detector system will not be 100$\%$ efficient, and may have less than optimal energy resolution. All of these factors combined may reduce the feasibility and/or utility of quantification. Additionally, these simulations were run generating six million particles. For high activity sources this can occur in a matter of seconds, but for lower activity sources, the time required to produce (1 kBq = 100 min) and detect that many particles may prove too long in a search and recovery scenario.

\subsection{Recommended Future Work}
\noindent In the future, more complexity should be added to the model to include:

\begin{itemize}
  \item Background radiation
  \item Less penetrating gammas
  \item True detector performance
\end{itemize}

Additionally this model should be modified to represent a real location and compared with real world results.

All geometry, simulation, parsing codes, analysis code, and perspective images for all simulations can be viewed at https://github.com/DVStepter/Attenuation.
